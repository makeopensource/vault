\documentclass{article}
\usepackage[utf8]{inputenc}
\usepackage{hyperref}
\usepackage{hypcap}

\title{Meeting Notes}
\author{MakeOpenSource, The University at Buffalo}
\date{2022-03-31}

\begin{document}

\maketitle

\tableofcontents
\clearpage

\section{Administrata}
\subsection{Email}
\begin{itemize}
    \item Forwarding all email received at \verb|opensrc@buffalo.edu| to each of the club staff members, so that they may respond without having to access the email itself.
    \item All messages sent to \verb|opensrc@buffalo.edu| should be responded to from an individual's university email address.
\end{itemize}
\par

\subsection{Planning meetings}
Planning events/meetings workflow:
\begin{enumerate}
    \item Planning all events \textit{\textbf{at least}} 2 weeks in advance.
    \item Deciding on multiple potential locations for said event.
    \item \textbf{Marking down the event} \textit{date} and \textit{time} on the shared calendar.
    \item applying for the event place and time on \verb|UBLinked|
    \item Announcing the event \textit{\textbf{after}} it is approved on \verb|UBLinked|.
\end{enumerate}
Consider more flexible and diverse meeting times.
\begin{enumerate}
    \item 2-3 meetings a week for different rotating groups.
\end{enumerate}

\subsection{Time management}
Attempt to better manage time in meetings to maximize progress.
\begin{itemize}
    \item Create more structured outlines and expected workflow for individual project management.
    \subitem Project Issues.
    \subitem Pull requests.
    \subitem 'Kanban' style project breakdown.
    \item Create an outline for each work session for division of time.
    \subitem e.g. devoting one hour per project focus.
\end{itemize}

\subsection {Development server}
\begin{itemize}
    \item Launch the on campus development server (Super Development Server).
    \item Look into raspberry pi clusters for distributed web applications.
    \item Launch the website rebuild on the current primary development server.
\end{itemize}

\section{Outreach}
\subsection{Career services}
\begin{itemize}
    \item Reach out to career services to gauge what resources and contacts they may have.
\end{itemize}
\subsection{Voltron analytics}
\begin{enumerate}
    \item Reach out to a contact at Voltron analytics to formulate a working plan for what they would like to work with us on, as well as time commitments.
    \subitem Apache Arrow
\end{enumerate}

\section{Current projects}
\subsection{Website}
\begin{enumerate}
    \item Deploy the rebuild version of the website onto the development server.
    \item Look into using Git for file hosting.
    \item Explore options for bulk email sending.
\end{enumerate}
\subsection{DevU}
\begin{enumerate}
    \item Have Jesse speak with club members to gauge interest of starting working on the autolab rework project.
    \item Recreate VODs from the previous work to help get members started.
    \item Set aside a portion of meetings to work on the project.
    \item Have a "setup your development environment" workshop.
\end{enumerate}
\clearpage
\section{Presentation series}
Bring back talks (either from us or outside people) about different diverse introductory topics.
\subsection{List of ideas for presentations}
\begin{enumerate}
    \item Git
    \item The *nix command line
    \item regular expressions (regex)
    \item "Learning the editor you don't want to learn", e.g. \verb|vim -> emacs| or \verb|emacs -> vim|
    \item How to put/phrase projects on your resume.
    \item Navigating internship opportunities.
    \item "Code reviews": what working in industry may look like.
\end{enumerate}

\section{Development "one shots"}
Have interactive workshops where the end goal is a finished (or mostly finished) product.
\subsection{List of ideas for development sessions}
\begin{enumerate}
    \item "Discord bot night": Creating a discord bot in python from the ground up using the \verb|py-cord| python framework.
    \item "Make a website": Using GitHub's free website hosting, walk through creating a personalized portfolio website.
\end{enumerate}

\end{document}
